\documentclass[12pt]{article}

\usepackage[utf8]{inputenc}
\usepackage[brazil]{babel}
\usepackage{amsmath,amssymb}
\usepackage{pdfsync}
\usepackage[all]{xy}
\usepackage{color}

\usepackage{graphicx}
\usepackage{url}
\usepackage{float}
\usepackage{listings}
\usepackage{todonotes}
\usepackage{algorithmic}
\usepackage{algorithm}
\usepackage{hyperref}

\title{{\large Universidade de Brasília \\ Instituto de Ciências Exatas \\
Departamento de Ciência da Computação} \\[1cm]
117536 - Lógica Computacional Turma: A\\[.5cm]
Relatório sobre {\bf Radix Sort}}
\author{Alberto Tavares Duarte Neto - 18/0011707}
\date{\today}

\begin{document}
\maketitle

\section{Introdução}

Este é o relatório do trabalho da disciplina Lógica Computacional 1, com o objetivo de estudar o assistente de provas PVS através da realização da verificação formal da corretude do algoritmo de ordenação Radixsort.

Para a implementação do radix sort é necessário outro algoritmo de ordenação. Neste trabalho foi utilizado o mergesort para tal, cujas algumas propriedades foram provadas em semestres anteriores por outras turmas.

\section{Apresentação do Problema e Solução}

radix sort é um algoritmo de ordenação que... (completar)

Em outras palavras, utilizando as definições dadas, devemos provar:

\begin{enumerate}
    \item permutations[T](merge\_sort(l), l)
    \item permutations[T](radixsort(l), l)
    \item is\_sorted?[T, lex](radixsort(l))
\end{enumerate}

onde l é uma lista do tipo abstrato T.

\subsection{Questão 1}

Nesta questão devemos provar que o merge sort é uma permutação da lista.

\subsubsection{Lemas Utilizados}

\begin{itemize}
    \item merge\_occurence
    \item length\_prefix
    \item length\_suffix
    \item app\_prefix\_sum
    \item occurrences\_of\_app
\end{itemize}

\subsubsection{Solução}

A estratégia de prova utilizada foi realizar uma indução no tamanho da lista, utilizando o fato de que, ao expandir a definição de merge sort,
temos o merge do prefixo e do sufixo da lista que são de fato menores que a lista, se encaixando na hipótese de indução.

Temos então o seguinte sequente ao expandir as definições de permutations e merge sort:

\begin{figure}[H]
    \centering
    \includegraphics[width=1 \textwidth]{seq1-1.png}
    \caption{}
    \label{}
\end{figure}

Onde x é uma lista com elementos do tipo T, e x\_1 é um elemento do tipo T.

O caso onde 

\begin{equation}
    length(x) <= 1
\end{equation}

é trivial, então trabalharemos o caso 

\begin{equation}
    length(x) > 1
\end{equation}

Pelo lema (botar numero) nós temos:

\begin{equation}
    occurrences(merge(l1, l2))(x) = occurences(l1)(x) + occurrences(l2)(x)
\end{equation}{}

Podemos instanciar este com o prefixo e suffixo de l nas listas l1 e l2, respectivamente, e substituir a igualdade no consequente.

Assim obtemos:

\begin{figure}[H]
    \centering
    \includegraphics[width=1 \textwidth]{seq1-2.png}
    \caption{}
    \label{}
\end{figure}

Neste sequente fica clara a utilização da hipótese de indução; é fácil
ver, utilizando os lemas (n e m), que para uma lista de tamanho maior que 1 seu prefixo e sufixo terão tamanho menor que a lista original.

Portanto, aplicando a hipótese indutiva ao prefixo e sufixo, temos que as ocorrências do merge sort do prefixo e sufixo são iguais às ocorrências do prefixo e sufixo, respectivamente, e aplicando isto ao consequente:

\begin{figure}[H]
    \centering
    \includegraphics[width=1 \textwidth]{seq1-3.png}
    \caption{}
    \label{}
\end{figure}

Por fim devemos utilizar dois lemas: (n) e (m).
Desta forma tempos que uma lista x é igual ao append de seu prefixo e sufixo; e as ocorrências de um append de duas listas l1 e l2 são iguais às da soma das ocorrências de l1 e l2.

\begin{figure}[H]
    \centering
    \includegraphics[width=1 \textwidth]{seq1-4.png}
    \caption{}
    \label{}
\end{figure}

Podemos aplicar estes dois fatos à equação no consequente, terminando a prova.


\subsection{Questão 2}

\subsubsection{Lemas Utilizados}

\begin{itemize}
    \item permutations\_is\_transitive
    \item merge\_sort\_is\_permutation
\end{itemize}

\subsubsection{Solução}

% sequente expansao radixsort

A prova desta questão é simples e segue diretamente da questão anterior. Utilizando o lema, agora provado, de que merge sort permuta a lista l nós podemos realizar duas instâncias deste:

\begin{figure}[H]
    \centering
    \includegraphics[width=1 \textwidth]{seq2-1.png}
    \caption{}
    \label{}
\end{figure}

Então utilizando a transitividade da permutação, que nos é garantida pelo lema (n), obtemos o resultado

\begin{equation}
    permutations(merge\_sort[T, <<](merge\_sort[T, <=](l)), l)
\end{equation}

que é exatamente o que queríamos provar.

\subsection{Questão 3}

Queremos mostrar que radixsort está ordenado segundo a ordem lexicográfica lex. A ideia da demonstração é utilizar o fato de que (...)

\subsubsection{Lemas Utilizados}

\begin{itemize}
    \item merge\_sort\_is\_sorted
    \item merge\_sort\_is\_conservative
    \item is\_sorted\_implies\_monotone
\end{itemize}

\subsubsection{Solução}

Após expandir as definições de radixsort, is\_sorted? e lex, temos:

\begin{figure}[H]
    \centering
    \includegraphics[width=1 \textwidth]{seq3-11.png}
    \caption{}
    \label{}
\end{figure}

ou seja, se k e 1 + k são índices da lista então os elementos com estes índices estão ordenados segundo a ordem lexicográfica.

Para proceder com a prova podemos utilizar o lema da ordenação do merge sort com a ordem <<; com isso podemos mostrar que, ao aplicar radix sort a uma lista, o elemento do índice k é menor que o elemento de índice 1 + k segundo esta ordem.

Portanto, aplicando simplificações proposicionais, obtemos:

\begin{figure}[H]
    \centering
    \includegraphics[width=1 \textwidth]{seq3-2.png}
    \caption{}
    \label{}
\end{figure}


A fórmula -3, que estava no consequente, foi para o antecedente negado após a simplificação proposicional, e com isso temos que os elementos de índice k e 1 + k são iguais segundo a ordem <<. 

Para que estes elementos estejam ordenados segundo a ordem <=, devemos ter que o merge sort é um algoritmo de ordenação estável; ou seja, se dois elementos são iguais então eles mantêm sua ordem relativa após a aplicação do algoritmo.

Utilizando o lema da estabilidade do merge sort temos, então:

\begin{figure}[H]
    \centering
    \includegraphics[width=1 \textwidth]{seq3-3.png}
    \caption{}
    \label{}
\end{figure}

É fácil de imaginar que, se i < j, o elemento de índice i é menor segundo << que o elemento de índice j após aplicarmos merge sort com a ordem <= na lista l. 

Isso é garantido por lema, de modo que este tem de hipótese que a lista l estar ordenada mas temos, também por lema, que a lista retornada pelo merge sort é ordenada.

Então a demonstração está concluída.


\section{Conclusão}



\end{document}

%%% Local Variables:
%%% mode: latex
%%% TeX-master: t
%%% End: